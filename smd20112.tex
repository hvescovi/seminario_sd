\documentclass[]{beamer}
% Class options include: notes, notesonly, handout, trans,
%                        hidesubsections, shadesubsections,
%                        inrow, blue, red, grey, brown

\usepackage{beamerthemesplit} 
% Other themes include: beamerthemebars, beamerthemelined, 
%                       beamerthemetree, beamerthemetreebars  
\usepackage[brazil]{babel}
\setbeamertemplate{footline}[frame number] 
\usepackage[utf8]{inputenc}

\title{Sistemas Multimídia Distribuidos}
\author{}                
\institute{}     
%\date{\today}         
\date{11/08/2011}

\begin{document}

% Creates title page of slide show using above information
\begin{frame}
  \titlepage
\end{frame}

%\section[Roteiro]{}

% Creates table of contents slide incorporating
% all \section and \subsection commands
%\begin{frame}
%  \tableofcontents
%\end{frame}

\begin{frame}
 \frametitle{Abstract}
 \begin{itemize}
   \item 
 \end{itemize}
\end{frame}

\section{Panorama geral}

\subsection{1.1 The complexity problem}

\begin{frame}
 \frametitle{Panorama}
 \begin{itemize}
   \item Fluxos contínuos de dados em tempo real
   \item Grandes quantidades de áudio, vídeo e outros elementos, respeitando o critério temporal
   \item Elementos de dados distribuídos com atrasos geralmente são eliminados
 \end{itemize}
\end{frame}

\begin{frame}
 \frametitle{Panorama II}
 \begin{itemize}
  \item Especificação: em termos de taxa de passagem de dados (largura de banda),
atraso da distribuição de cada elemento (latência) e taxa de eliminação/perca de pacotes
  \item Latência: especialmente importante em aplicativos interativos
  \item Perca de pacotes é aceitável quando é possível re-sincronizar após o ponto de perda
\end{itemize}
\end{frame}

\begin{frame}
  \frametitle{Panorama III}
\begin{itemize}
  \item Alocação de recursos é referida como qualidade de serviços: alocação de processamento,
largura de banda da rede e memória (para buffer)
\end{itemize}
\end{frame}

\subsection{Introdução}

\begin{frame}
  \frametitle{Introdução}
\begin{itemize}
  \item Fluxos de dados contínuos (streams) baseados no tempo: telefonia pela Internet, 
vídeoconferência, etc
  \item A qualidade geral é ruim; é imprópria para: TV digital/interativa, supervisão com vídeo
  \item Sistemas multimídia são sistemas em tempo real: precisam executar tarefas e apresentar
resultados de acordo com um escalonamento determinado externamente
  \item O grau de sucesso desse fornecimento é o QoS (Quality of Service), usufruída
pelo aplicativo
\end{itemize}
\end{frame}

\begin{frame}
  \frametitle{Introdução}
\begin{itemize}
  \item Diferenças entre os sistemas de tempo real de aviação, processo de fabricação, etc: 
  \item estes possuem volumes de dados pequenos e prazos finais rígidos; o não cumprimento pode ter
consequências desastrosas, por isso superestima-se recursos e trabalha-se com atendimento 
no pior caso
  \item os sistemas multimídia:
  \item operam dentro de um ambiente geral, competindo com recursos e banda de rede com
 outros aplicativos distribuídos
  \item os requisitos são dinâmicos: mais participantes, mais recursos necessários; ou uma 
simulação pode requerer mais processamento
  \item operação de sistemas multimídia em conjunto com outras aplicações: edição de textos, 
conversa de voz separada, mensagens instantâneas, em meio a uma vídeo-conferência
\end{itemize}
\end{frame}

\begin{frame}
  \frametitle{Introdução}
  Serviços providos por um sistema distribuído típico:
  \begin{itemize}
    \item vídeoconferência em desktop
    \item acesso a sequência de vídeo
    \item transmissão de TV e rádio digital
  \end{itemize}
  Recursos para o gerenciamento da qualidade de serviço: largura de banda da rede, ciclos do
processador e capacidade de memória
\end{frame}

\begin{frame}
  \frametitle{Introdução}
\begin{itemize}
  \item FIGURAAAAAAAAAAA 17.1
\end{itemize}
\end{frame}

\begin{frame}
  \frametitle{Introdução}
\begin{itemize}
  \item Sistema distribuído aberto: aplicativos multimídia podem ser iniciados sem organização
anterior\footnote{O QUE ISSO QUER DIZER EXATAMENTE??} e coexistir na mesma rede
  \item É necessário haver qualidade do serviço independentemente da qualidade total do sistema
\end{itemize}
\end{frame}

\begin{frame}
  \frametitle{Introdução}
Aplicativos multimídia que têm sido implantados:
\begin{itemize}
  \item Multimídia baseada na web: permite acesso aos fluxos de áudio e vídeo na Web; buffers
podem fornecer exibição contínua e suave mas com atraso da origem para o destino (segundos)
  \item Telefone de rede e áudio-conferência: aplicações de natureza interativa com baixos
atrasos de RTT\footnote{round-trip time, tempo de ida e volta}
  \item Vídeo sob demanda: largura de banda, servidor de vídeo e estações, todos dedicados;
alto uso de buffers no destino
\end{itemize}
\end{frame}

\begin{frame}
  \frametitle{Introdução}
Aplicativos muito interativos: problemas...
\begin{itemize}
  \item telefonia na Internet - VOIP
  \item vídeoconferência: restrições de largura de banda e latência\footnote{VER CONCEITO DISSO}
  \item ensaio de execução musical distribuída: severas restrições de sincronização
\end{itemize}
\end{frame}

\begin{frame}
  \frametitle{Introdução}
Exigências das aplicações super-interativas
\begin{itemize}
  \item comunicação com baixa latência: RTT de 100 a 300 ms
  \item estado distribuído síncrono: se um usuário interrompe um vídeo, todos devem 
ver a interrupção no mesmo quadro
  \item sincronismo de mídia: o exemplo da execução musical distribuída; 
Konstantas et al. [1997] aponta até 50ms; fluxos separados de áudio e vídeo devem manter
sincronismo \emph{labial}\footnote{exemplo: sessão de karaokê distribuída}
  \item sincronização externa: aplicações cooperativas diversas devem parecer 
sincronizadas\footnote{Isso é perceptível quando \emph{filmamos a televisão}}
com os fluxos multimídia baseados no tempo (exemplo: animações de computador, dados CAD, 
quadros-negros eletrônicos).
\end{itemize}
\end{frame}

\begin{frame}
  \frametitle{Introdução}
Janela de escassez
\begin{itemize}
  \item Sistemas atuais tem capacidade para manipular dados multimídia
  \item As limitações estão nos recursos necessários, especialmente 
na quantidade e qualidade de fornecimento de fluxos
  \item É necessário alocar e escalonar os recursos
  \item \emph{Antes que a janela de escassez seja alcançada, 
um sistema tem recursos insuficientes para executar as aplicações relevantes}
\end{itemize}
\end{frame}

\begin{frame}
  \frametitle{Introdução}
\begin{itemize}
  \item FIGURA DA JANELA DE ESCASSEZ
\end{itemize}
\end{frame}

\subsection{Características dos dados multimídia}

\begin{frame}
  \frametitle{Características}
Algumas definições
\begin{itemize}
  \item mídia contínua é uma sequência de valores discretos que substituem-se uns aos outros
com o passar do tempo; exemplo: uma imagem é amostrada 25 vezes 
por segundo para dar impressão de movimento com qualidade de TV; um sinal sonoro é amostrado
8000 vezes por segundo para transmitir fala com a qualidade de um telefone
  \item os fluxos multimídia são baseados no tempo, ou isocrônicos: os tempos nos quais
os valores são reproduzidos ou gravados afetam a validade dos dados, definem 
a \emph{semântica} ou conteúdo do fluxo
  \item dados multimídia são volumosos: precisam de maior desempenho de entrada/saída que os
sistemas convencionais
\end{itemize}
\end{frame}

\begin{frame}
  \frametitle{Características}
\begin{itemize}
  \item FIGURA QUADRO TAXAS E AMOSTRAS DE DADOS
\end{itemize}
\end{frame}

\begin{frame}
  \frametitle{Características}
\begin{itemize}
  \item 
\end{itemize}
\end{frame}

\begin{frame}
  \frametitle{Características}
\begin{itemize}
  \item 
\end{itemize}
\end{frame}

\begin{frame}
  \frametitle{Características}
\begin{itemize}
  \item 
\end{itemize}
\end{frame}

\begin{frame}
  \frametitle{Características}
\begin{itemize}
  \item 
\end{itemize}
\end{frame}

\begin{frame}
  \frametitle{Características}
\begin{itemize}
  \item 
\end{itemize}
\end{frame}

\begin{frame}
  \frametitle{Características}
\begin{itemize}
  \item 
\end{itemize}
\end{frame}

%\begin{frame}
% \frametitle{}
% \begin{itemize}
%   \item 
%  \end{itemize}
% \end{frame}



%\begin{frame}
% \frametitle{Questionário, exemplo I}
%  \begin{figure}[hbtp]
%  \begin{center}
%   \includegraphics[scale=0.28]{imagens/questionario_exemplo1.png}
%  \end{center}
% \end{figure}
%\end{frame}


\begin{frame}
 \frametitle{FIM}
   FIM% - Obrigado
\end{frame}

%\section{Referências}

%\bibliographystyle{plain}
%\bibliography{artigosdois}

\end{document}