\documentclass[]{beamer}
% Class options include: notes, notesonly, handout, trans,
%                        hidesubsections, shadesubsections,
%                        inrow, blue, red, grey, brown

\usepackage{beamerthemesplit} 
% Other themes include: beamerthemebars, beamerthemelined, 
%                       beamerthemetree, beamerthemetreebars  
\usepackage[brazil]{babel}
\setbeamertemplate{footline}[frame number] 
\usepackage[utf8]{inputenc}

\title{Sistemas Multimídia Distribuidos}
\author{}                
\institute{}     
%\date{\today}         
\date{11/08/2011}

\begin{document}

% Creates title page of slide show using above information
\begin{frame}
  \titlepage
\end{frame}

%\section[Roteiro]{}

% Creates table of contents slide incorporating
% all \section and \subsection commands
%\begin{frame}
%  \tableofcontents
%\end{frame}

\begin{frame}
 \frametitle{Abstract}
 \begin{itemize}
   \item 
 \end{itemize}
\end{frame}

\section{Panorama geral}

\subsection{1.1 The complexity problem}

\begin{frame}
 \frametitle{Panorama}
 \begin{itemize}
   \item Fluxos contínuos de dados em tempo real
   \item Grandes quantidades de áudio, vídeo e outros elementos, respeitando o critério temporal
   \item Elementos de dados distribuídos com atrasos geralmente são eliminados
 \end{itemize}
\end{frame}

\begin{frame}
 \frametitle{Panorama II}
 \begin{itemize}
  \item Especificação: em termos de taxa de passagem de dados (largura de banda),
atraso da distribuição de cada elemento (latência) e taxa de eliminação/perca de pacotes
  \item Latência: especialmente importante em aplicativos interativos
  \item Perca de pacotes é aceitável quando é possível re-sincronizar após o ponto de perda
\end{itemize}
\end{frame}

\begin{frame}
  \frametitle{Panorama III}
\begin{itemize}
  \item Alocação de recursos é referida como qualidade de serviços: alocação de processamento,
largura de banda da rede e memória (para buffer)
\end{itemize}
\end{frame}

\subsection{Introdução}

\begin{frame}
  \frametitle{Introdução}
\begin{itemize}
  \item Fluxos de dados contínuos (streams) baseados no tempo: telefonia pela Internet, 
vídeoconferência, etc
  \item A qualidade geral é ruim; é imprópria para: TV digital/interativa, supervisão com vídeo
  \item Sistemas multimídia são sistemas em tempo real: precisam executar tarefas e apresentar
resultados de acordo com um escalonamento determinado externamente
  \item O grau de sucesso desse fornecimento é o QoS (Quality of Service), usufruída
pelo aplicativo
\end{itemize}
\end{frame}

\begin{frame}
  \frametitle{Introdução}
\begin{itemize}
  \item Diferenças entre os sistemas de tempo real de aviação, processo de fabricação, etc: 
  \item estes possuem volumes de dados pequenos e prazos finais rígidos; o não cumprimento pode ter
consequências desastrosas, por isso superestima-se recursos e trabalha-se com atendimento 
no pior caso
  \item os sistemas multimídia:
  \item operam dentro de um ambiente geral, competindo com recursos e banda de rede com
 outros aplicativos distribuídos
  \item os requisitos são dinâmicos: mais participantes, mais recursos necessários; ou uma 
simulação pode requerer mais processamento
  \item operação de sistemas multimídia em conjunto com outras aplicações: edição de textos, 
conversa de voz separada, mensagens instantâneas, em meio a uma vídeo-conferência
\end{itemize}
\end{frame}

\begin{frame}
  \frametitle{Introdução}
  Serviços providos por um sistema distribuído típico:
  \begin{itemize}
    \item vídeoconferência em desktop
    \item acesso a sequência de vídeo
    \item transmissão de TV e rádio digital
  \end{itemize}
  Recursos para o gerenciamento da qualidade de serviço: largura de banda da rede, ciclos do
processador e capacidade de memória
\end{frame}

\begin{frame}
  \frametitle{Introdução}
\begin{itemize}
  \item FIGURAAAAAAAAAAA 17.1
\end{itemize}
\end{frame}

\begin{frame}
  \frametitle{Introdução}
\begin{itemize}
  \item Sistema distribuído aberto: aplicativos multimídia podem ser iniciados sem organização
anterior\footnote{O QUE ISSO QUER DIZER EXATAMENTE??} e coexistir na mesma rede
  \item É necessário haver qualidade do serviço independentemente da qualidade total do sistema
\end{itemize}
\end{frame}

\begin{frame}
  \frametitle{Introdução}
Aplicativos multimídia que têm sido implantados:
\begin{itemize}
  \item Multimídia baseada na web: permite acesso aos fluxos de áudio e vídeo na Web; buffers
podem fornecer exibição contínua e suave mas com atraso da origem para o destino (segundos)
  \item Telefone de rede e áudio-conferência: aplicações de natureza interativa com baixos
atrasos de RTT\footnote{round-trip time, tempo de ida e volta}
  \item Vídeo sob demanda: largura de banda, servidor de vídeo e estações, todos dedicados;
alto uso de buffers no destino
\end{itemize}
\end{frame}

\begin{frame}
  \frametitle{Introdução}
Aplicativos muito interativos: problemas...
\begin{itemize}
  \item telefonia na Internet - VOIP
  \item vídeoconferência: restrições de largura de banda e latência\footnote{VER CONCEITO DISSO}
  \item ensaio de execução musical distribuída: severas restrições de sincronização
\end{itemize}
\end{frame}

\begin{frame}
  \frametitle{Introdução}
Exigências das aplicações super-interativas
\begin{itemize}
  \item comunicação com baixa latência: RTT de 100 a 300 ms
  \item estado distribuído síncrono: se um usuário interrompe um vídeo, todos devem 
ver a interrupção no mesmo quadro
  \item sincronismo de mídia: o exemplo da execução musical distribuída; 
Konstantas et al. [1997] aponta até 50ms; fluxos separados de áudio e vídeo devem manter
sincronismo \emph{labial}\footnote{exemplo: sessão de karaokê distribuída}
  \item sincronização externa: aplicações cooperativas diversas devem parecer 
sincronizadas\footnote{Isso é perceptível quando \emph{filmamos a televisão}}
com os fluxos multimídia baseados no tempo (exemplo: animações de computador, dados CAD, 
quadros-negros eletrônicos).
\end{itemize}
\end{frame}

\begin{frame}
  \frametitle{Introdução}
Janela de escassez
\begin{itemize}
  \item Sistemas atuais tem capacidade para manipular dados multimídia
  \item As limitações estão nos recursos necessários, especialmente 
na quantidade e qualidade de fornecimento de fluxos
  \item É necessário alocar e escalonar os recursos
  \item \emph{Antes que a janela de escassez seja alcançada, 
um sistema tem recursos insuficientes para executar as aplicações relevantes}
\end{itemize}
\end{frame}

\begin{frame}
  \frametitle{Introdução}
\begin{itemize}
  \item FIGURA DA JANELA DE ESCASSEZ
\end{itemize}
\end{frame}

\subsection{Características dos dados multimídia}

\begin{frame}
  \frametitle{Características}
Algumas definições
\begin{itemize}
  \item mídia contínua: sequência de valores discretos que substituem-se uns aos outros
com o passar do tempo; ex: uma imagem é amostrada 25 vezes/seg. para 
dar impressão de movimento com qualidade de TV; um sinal sonoro é amostrado
8000 vezes/seg para transmitir fala com a qualidade de um telefone
  \item fluxos multimídia são baseados no tempo\footnote{ou isocrônicos}: os tempos nos quais
os valores são reproduzidos ou gravados afetam a validade dos dados, definem 
a \emph{semântica} ou conteúdo do fluxo
  
\end{itemize}
\end{frame}

\begin{frame}
  \frametitle{Características}
\begin{itemize}
  \item FIGURA QUADRO TAXAS E AMOSTRAS DE DADOS
\end{itemize}
\end{frame}

\begin{frame}
  \frametitle{Características}
\begin{itemize}
\item dados multimídia são volumosos: precisam de maior desempenho de entrada/saída que os
sistemas convencionais
  \item utiliza-se compactação, embora transformações, como mistura de vídeo, sejam 
difíceis de realizar com fluxos compactados.
  \item compactação pode reduzir requisitos de largura de banda, mas não requisitos de
temporização de dados contínuos
  \item codecs em hardware podem realizar a compactação; codecs em software oferecem
maior flexibilidade
\end{itemize}
\end{frame}

\begin{frame}
  \frametitle{Características}
\begin{itemize}
  \item método MPEG é assimétrico, com algoritmo de compactação complexo e descompactação
simples
  \item isso ajuda em conferências no desktop: compactação feita por codec em hardware, 
descompactação via software, ????permitindo que o número de participantes na
vídeoconferência varie sem considerar o número de 
codecs no computador de cada usuário???? NÃO ENTENDI ESSA FRASE
\end{itemize}
\end{frame}

\subsection{Gerenciamento de qualidade de serviço}

\begin{frame}
  \frametitle{Gerenciamento da Qualidade de Serviço}
\begin{itemize}
  \item aplicações multimídia executadas em redes de PC's competem por recursos: ciclos 
de processador, barramento, capacidade de buffer)
  \item redes são projetadas pra que mensagens de diferentes origens sejam intercaladas,
permitindo a existência de muitos canais de comunicação virtuais nos mesmos canais físicos
  \item Ethernet: gerencia um meio de transmissão compartilhado na base do melhor esforço
(QUE BUDEGA É ESSA! apenas não-confiável?)
  \item enfim: rodízio, tempos aleatórios, etc, podem não satisfazer as necessidades das
aplicações multimídia: distribuição atrasada não tem valor.
\end{itemize}
\end{frame}

\begin{frame}
FIGURA 17.4: arquitetura abstrata com fluxos de mídia de dados gerados continuamente
\end{frame}

\begin{frame}
FIGURA 17.5: requisitos de recurso para a fig 17.4
\end{frame}

\begin{frame}
Principais responsabilidades do gerenciador de qualidade do serviço:
FIGURA 17.6: a tarefa do gerenciador de qualidade do serviço
\end{frame}


\begin{frame}
  \frametitle{QoS: Negociação}
A aplicação especifica requisitos de qualidade através de 3 parâmetros: 
\begin{itemize}
  \item largura de banda: a taxa na qual os dados fluem pelo fluxo
  \item latência: tempo exigido para um elemento de dados individual se mover
em um fluxo, da origem até o destino; a variação dessa latência é denominada \emph{jitter}
  \item taxa de perda: quadros de vídeo ou amostras de áudio eliminados; até 1\%; 
para aplicações críticas, bem menos!
\end{itemize}
\end{frame}

\begin{frame}
 
FIGURA do grafico do JITTER
jitter.jpg
referência: http://www.ipg.pt/user/~sduarte/rc/Trabalhos2005/QoS/necessidadesQoS.htm

\end{frame}


\begin{frame}
  \frametitle{QoS: Negociação}
Exemplos de descrição dos parâmetros: 
\begin{itemize}
 \item descrevendo as características de um fluxo: numa aplicação de conferência,
é preciso largura de banda média de 1,5Mbps; atraso máximo
de 150ms para evitar hiatos na palestra; o algoritmo de descompactação
no destino pode produzir imagens aceitáveis com uma perda de 1 quadro em 100
  \item descrevendo capacidade de fluxo: uma rede pode fornecer conexões de largura de banda
de 64kpbs; os algoritmos de enfileiramento garantem atrasos de menos de 10ms; o sistema
de transmissão garante uma taxa de perda menor que 1 em 10e6.
\end{itemize}
\end{frame}

\begin{frame}
  \frametitle{QoS: Negociação}
Os parâmetros são interdependentes. Exemplos:
\begin{itemize}
  \item taxa de perda depende de estouro do buffer e dados dependentes do tempo
chegando atrasados; logo, quando maior a largura de banda e o atraso, menor será
a taxa de perda
  \item quanto menor a largura de banda, em relação à carga, 
mais mensagens serão armazenadas e mais buffers serão necessários; quanto maior o buffer,
mais provável que mensagens esperem outras que estão na frente para serem atendidas, e
assim, maior será o atraso. ESTUDAR ESSA PARTE!!!
\end{itemize}
\end{frame}

\begin{frame}
  \frametitle{QoS}
Especificando parâmetros de QoS: largura de banda
\begin{itemize}
  \item  para MPEG, a compactação média está entre 1:50 e 1:100, a depender
do conteúdo; logo, parâmetros de qualidade são citados como valores mínimo, médio e máximo.
  \item taxa de rajada\footnote{Uma CPU transfere dados via canais ou barramento 
de duas maneiras, byte a byte ou então em blocos de bytes de cada vez; 
é esta modalidade de transferência em blocos é chamada de 'modo rajada'.\cite{site1:2011}} 
(\emph{burst}): considere 3 fluxos de 1Mbps: o primeiro transfere um único quadro de 1Mbit/s;
o segundo é um fluxo assíncrono de elementos de animação
com largura de banda média de 1Mbps; o terceiro envia amostra de 
som de 100bits a cada microssegundo. Os 3 fluxos exigem a mesma largura de banda, porém
seus padrões de tráfego são diferentes. O parâmetro de rajada define o número máximo de 
elementos de mídia que podem chegar cedo, isto é, antes do que devam chegar, de acordo
com a taxa normal. ESTUDAR MAIS SOBRE RAJADA!!
\end{itemize}
\end{frame}

\begin{frame}
  \frametitle{QoS}
\begin{itemize}
  \item Anderson
%\cite{Anderson:1993} BAIXAR: Meta-scheduling for distributed continuous media, em ACM
define o número máximo de mensagens em um fluxo durante qualquer intervalo t como Rt+B,
onde R é a taxa e B é o tamanho máximo da rajada.
  \item reflete bem dados multimídia: dados multimídia lidos do disco são geralmente 
distribuídos em blocos grandes,
e recebidos das redes em pacotes pequenos; o parâmetro de rajada define a quantidade de
espaço exigida no buffer para evitar perda.
\end{itemize}
\end{frame}

\begin{frame}
  \frametitle{QoS}
\begin{itemize}
  \item Latência: PAREI na página 630)
\end{itemize}
\end{frame}

\begin{frame}
  \frametitle{QoS}
\begin{itemize}
  \item 
\end{itemize}
\end{frame}

\begin{frame}
  \frametitle{QoS}
\begin{itemize}
  \item 
\end{itemize}
\end{frame}

\begin{frame}
  \frametitle{QoS}
\begin{itemize}
  \item 
\end{itemize}
\end{frame}

%\begin{frame}
% \frametitle{}
% \begin{itemize}
%   \item 
%  \end{itemize}
% \end{frame}



%\begin{frame}
% \frametitle{Questionário, exemplo I}
%  \begin{figure}[hbtp]
%  \begin{center}
%   \includegraphics[scale=0.28]{imagens/questionario_exemplo1.png}
%  \end{center}
% \end{figure}
%\end{frame}


\begin{frame}
 \frametitle{FIM}
   FIM% - Obrigado
\end{frame}

\section{Referências}

%\bibliographystyle{plain}
\bibliography{refs}

\end{document}